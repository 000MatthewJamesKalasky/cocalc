\documentclass{article}

%%%%%%%%%%%%%%%%%%%%%%%%%%%%%%%%%%%%%%%%%%%%%%%%%%%%%%%%%%%%%%%%%%%%%%%%%%%%%%%%%%%%%%%%%%%%%
%
% Hello, this is the start of a LaTeX document
%
% First rule: everything after a percent-sign is a comment.
% It is ignored and does not show up in the final PDF document.
%
% Second rule: everything that starts with a backslash \ is a command.
% This is used for everything!
% * \begin{document} signals where the real meat of the file starts (after the preemble)
% * \section and \subsection structure the document
% * and much more ...
%
% Third rule: In between the commands, or inside the command's arguments in {...} you write text.
% You can make arbitrary single linebreaks, which will not break apart a paragraph.
% Two or more linebreaks signal the start of a new paragraph.
%
% Formulas: LaTeX is most famous for writing formulas. They're inside of $...$ or
% \begin{equation}
%  ...
% \end{equation}
% environments.
%
% Why? Well, this might seem overly complex, but in the end it is also very precise.
% You can define exactly what each part of the document should be and LaTeX formats it for you.
%
% Got interested? Here is a link to an online LaTeX book explaining more about all this:
% https://en.wikibooks.org/wiki/LaTeX
%
% Now, skim over the preemble and find the \begin{document} command
%
%%%%%%%%%%%%%%%%%%%%%%%%%%%%%%%%%%%%%%%%%%%%%%%%%%%%%%%%%%%%%%%%%%%%%%%%%%%%%%%%%%%%%%%%%%%%%

% set font encoding for PDFLaTeX or XeLaTeX
\usepackage{ifxetex}
\ifxetex
  \usepackage{fontspec}
\else
  \usepackage[T1]{fontenc}
  \usepackage[utf8]{inputenc}
  \usepackage{lmodern}
\fi

% used in maketitle
\title{My first \LaTeX{} document}
\author{My Name}

% Enable SageTeX to run SageMath code right inside this LaTeX file.
% documentation: http://mirrors.ctan.org/macros/latex/contrib/sagetex/sagetexpackage.pdf
\usepackage{sagetex}


%%%%%%%%%%%%%%%%%%%%%%%%%%%%%%%%%%%%%%%%%%%%%%%%%%%%%%%%%%%%%%%%%%%%%%%%%%%%%%%%%%%%%%%%%%%%%
\begin{document}

\section{First Steps}

This is a short introduction document to \LaTeX{}.
You write the code on the left hand side of this editor,
while compiled and rendered output shows up on the right hand side.

Pay attention to any syntax errors!
\LaTeX{} is a programming language and stops compiling and updating the document upon errors.
Any command starts with a backslash.
Watch out for the red "Errors" tab and inspect any problems.

\subsection{Subsection}

You can organize your document in sections,
just like this one here is a subsection!

\subsection{Another one ...}

Here is \textit{another one},
where the text ``another one'' is formatted italic.

\textbf{Bold font} is also possible.

\subsection{Formulas}

Latex is famous for setting formulas.
This is usually done between dollar signs.
For example: $\int_{x=0}^{\infty} \frac{1}{2 + x^2}\;\mathrm{d}x$.

\section{SageMath}

You can also run a couple of computations with Sage and embed the output right here in the document.

% this line here is an invisble comment
% and the block below some sage code setting x and y but not producing any output.
\begin{sagesilent}
x = var('x')
a = 1328782374
b = 2394728347628374
\end{sagesilent}

% Here, we use x and y defined above:
The product of $\sage{a}$ and $\sage{b}$  is \\
$\sage{a*b}$
and its prime factorization: $\sage{factor(a*b)}$.

It is also possible to create a plot:

\begin{center}
\sageplot[width=.5\textwidth]{plot(sin(x) * cos(3*x), (x, -10, 10))}
\end{center}

% this is the last command of the document.
\end{document}

%  Do NOT edit past the last command!

